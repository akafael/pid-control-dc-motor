% Pacotes e configurações padrão do estilo ``article''\
% -------------------------------------
\documentclass[a4paper,11pt]{article}
% Layout
% --------------------------------------------------------------------------------
%     Gráficos e layout ----------------------------------------------------------------------

\ifx\pdfmatch\undefined
\else
    \usepackage[T1]{fontenc}
    \usepackage[utf8]{inputenc}
\fi
% xetex:
\ifx\XeTeXinterchartoks\undefined
\else
    \usepackage{fontspec}
    \defaultfontfeatures{Ligatures=TeX}
\fi
% luatex:
\ifx\directlua\undefined
\else
    \usepackage{fontspec}
\fi
% End engine-specific settings

%      Fonte --------------------------------------------------------------------------------
%\usepackage{lmodern}
\usepackage{times}
%     Pacotes adicionados -------------------------------------------------------------------
\usepackage{ae}
%     Língua e hifenização ------------------------------------------------------------------
\usepackage[portuguese]{babel}
\usepackage{hyphenat}
%      Outros --------------------------------------------------------------------------------
\usepackage{hyperref} % Permite Links personalisados usando hyperref
\usepackage{fancyhdr}
\usepackage{sectsty}
\usepackage{float}   % Gerencia melhor o posicionamento das figuras e tabelas
\usepackage{array}
%\usepackage{graphicx}
\usepackage[pdftex]{color,graphicx}
\usepackage{hyperref}
\usepackage{enumerate} % Permite alterar Layout do enumerate
%\usepackage{pdflscape}  % Permite alterar a orientação da pagina para Paisagem
%\usepackage{ifthen}  % Permite usar condicionais ifelse
%\usepackage[table]{xcolor} % Permite alterar as cores das células de uma tabela
\usepackage{amsmath,amssymb} % Ambiente para uso de elementos matemáticos
\usepackage{caption}
\usepackage{subcaption} % permite o uso de multiplas figuras com legenda (ambiente subfigure)
\usepackage{minted} % Ambiente minted para colorir código de programas
\usepackage{natbib} % Para referencia bibliográfica
\usepackage{url}    % Referência de links na internet
%\usepackage{listings} % pacote para apresentar código de programação
\usepackage{indentfirst}  % Para indentar o primeiro parágrafo de cada seção
\usepackage{titling}  % Permite Montar uma página de titulo própria
\usepackage{python} % Permite usar python para programar (Pacote não oficial)

% Layout do documento ------------------------------------------------------------------------
%     Bordas e tamanho da página ------------------------------------------------------------
\usepackage{geometry} 
 \geometry{ % Padrõa ABNT para relatórios
 a4paper,
 left=30mm,
 right=20mm,
 top=30mm,
 bottom=20mm
 }
%     Cabeçalho e Rodapé ---------------------------------------------------------------
\pagestyle{fancy}
  \lhead{}
  \chead{}
  \rhead{}
  \lfoot{}
  \cfoot{}
  \rfoot{\thepage}
%     Númeração ------------------------------------------------------------------------
  \pagenumbering{arabic}
%     Retas do cabeçalho e rodapé ------------------------------------------------------
  \renewcommand{\headrulewidth}{0.5pt}
  \renewcommand{\footrulewidth}{0.5pt}
%     Tamanho da letra de seções e derivadas --------------------------------------------
  \sectionfont{\normalsize}
  \subsectionfont{\small}
%     Hiperlinks ------------------------------------------------------------------------
  \hypersetup{
                  colorlinks,
                  citecolor=black,
                  filecolor=black,
                  linkcolor=black,
                  urlcolor=black
                  }
%     Definições do pdf ----------------------------------------------------------------------
\hypersetup{
    unicode=false,          % non-Latin characters in Acrobat’s bookmarks
    pdftoolbar=true,        % show Acrobat’s toolbar?
    pdfmenubar=true,        % show Acrobat’s menu?
    pdffitwindow=false,     % window fit to page when opened
    pdfstartview={FitH},    % fits the width of the page to the window    
    pdfauthor={Rafael Lima},     % author
    pdfnewwindow=true      % links in new window
}
%     Outros ----------------------------------------------------------------------------
      %\renewcommand{\thesection}{(\alph{section})} % muda o estilo de númeração das sections
      % alterando a formatação dos numeradores de lista de itens
      \renewcommand\theenumi{\arabic{enumi}}
      \renewcommand\labelenumi{(\textit{\theenumi})}
	  \renewcommand\theenumii{\arabic{enumii}}
	  \renewcommand\labelenumii{(\textit{\theenumi.\theenumii})}
      
% ---------------------------------------------------------------------------------------


%\title{Laboratório X} % Define o título do Relatório
\author{Rafael Lima}

% Definições Auxiliares ( Macros próprias )
% -----------------------------------------------------------------
%\input{relat_aux.tex} % Arquivo com minhas macros
% ----------------------------------~>ø<~---------------------------------------
\begin{document}
% Capa e Índice ---------------------------------------------------------------
%--------------------------------------------------- Capa --------------------------------------------
%\newpage
\begin{figure}[h!]
\centering
\includegraphics[scale=0.9]{img/simb_unb.png}
\label{fig:unb}
\end{figure}

\begin{center}
{\LARGE Universidade de Brasília}\\
Departamento de Engenharia Elétrica (ENE)\\
Professor: Geovany A. Borges\\
Disciplina: Laboratório de Controle Dinâmico\\
\end{center}


\vspace{0.18\textheight}

\begin{center}
    \Huge \textbf{Laboratório de Controle Dinâmico \\}
\end{center}

\vspace*{\fill} % Completa espaço em branco e empurra o resto para o final da página

% Tabela com os nome das pessoas do grupo

\begin{table}[H]
    \begin{tabular}{ll}
        % Nome      & Matrícula
        Tiago Pereira Neves & 16/0077907 \\
%        Jefferson C. T. O. Lima & 18/0137107 \\
        Rafael Lima & 10/0131093 \\
    \end{tabular}
\end{table}

\vspace{0.5cm}

\begin{center}
    \textbf{Brasília\\
    \the\year} % Coloca o Ano atual
\end{center}

\thispagestyle{empty} % Retira o cabeçalho e o rodapé da página

% ------------------------------------------------- Índice -------------------------------------------
\newpage
\tableofcontents
\newpage
% ----------------------------------------------------------------------------------------------------

 % Capa para UnB
% Conteúdo -------------------------------------------------------------------

% Material de referência - prof
% https://drive.google.com/drive/folders/1IDdc7QVsdYIASgZzNlUelhGF_FaV2_EJ
\section{Resumo}

Neste relatório é aborda a problemática de controle de um sistema rotativo da \textit{Quanser} modelo \textit{SRV02}. Com base na teoria de engenharia de controle, são apresentados os passos para identificação do modelo, incluindo suas não-linearidades. Após identificação completa do modelo são propostos dois controles de diferentes abordagens para o controle de posição angular do sistema, o primeiro é realizado com controle Proporcional-Integrativo-Derivativo (PID) e o segundo com modelo em Espaço de Estados.

\section{Introdução}

A necessidade de utilização de controle para controlar um sistema e obter um comportamento desejado se estende à vários campos da sociedade, que vai desde um ascender e apagar de luzes, à um controle de movimento de um braço robótico.
Para realizar eficientemente o controle de um determinado sistema de acordo com o que se deseja, existem diferentes maneiras.
E neste trabalho, serão apresentadas duas formas de controle para utilização no controle de posição de um sistema rotativo.

% não precisa, temos até 18h para entregar

\section{Metodologia}

Neste trabalho o problema de controle de um sistema é abordado de maneira completa. Desde a identificação, modelagem e controle é assumido que não é conhecida a dinâmica do sistema, ainda que a empresa fabricante tenha fornecido o \textit{Workbook} \cite{studentWorkbook} do \textit{Kit} deste estudo. Desta forma o trabalho se aproxima do que o Engenheiro se deparará muitas vezes no mercado de trabalho, quando não se sabe nenhum parâmetro do sistema a ser controlado. Assim, o projeto foi dividido em três Fases ou etapas: 1ª - identificação e modelagem. 2ª - desenvolvimento de um controle PID. 3ª - desenvolvimento de um controle em espaço de estados.

Para identificar todos os parâmetros da função de transferência que rege a dinâmica de um sistema e possíveis não-linearidades que afetam essa dinâmica foi utilizada uma planta genérica afim de criar uma metodologia válida para ser aplicada no sistema real. Após validar a metodologia de identificação foi possível aplicar ao sistema real e obter seus parâmetros.

Com a identificação finalizada foi iniciada a parte de controle. A partir da função de transferência do sistema o controle PID foi desenvolvido utilizando a ferramenta \textit{Sisotool} do \textit{MATLAB}. E transformando a função de transferência para espaço de estados, foi também desenvolvido o controle para o sistema em espaço de estados.

Para gestão do projeto e de suas fases foram utilizadas as plataformas \textit{GitHub} como repositório, \textit{WhatsApp} para comunicação entre os membros do trabalho e \textit{Overleaf} para a escrita deste relatório. 

% http://ctms.engin.umich.edu/CTMS/index.php?example=MotorSpeed&section=SystemModeling
% http://ctms.engin.umich.edu/CTMS/index.php?example=MotorSpeed&section=SimulinkModeling
% http://www.inf.fu-berlin.de/lehre/WS04/Robotik/motors.pdf
% https://ir.nctu.edu.tw/bitstream/11536/125183/36/509-2.pdf
% http://users.isr.ist.utl.pt/~alex/micd0506/motordc.pdf
% http://homepages.laas.fr/lzaccari/seminars/DCmotors.pdf
% https://www.control.isy.liu.se/student/tsrt21/file/pm_dcmotor.pdf

% Técnica antiga de controle ------------------------
% https://en.wikipedia.org/wiki/Ward_Leonard_control

\section{Identificação e Modelagem}

% http://www.eletrica.ufpr.br/~gustavo/controle_digital/Intro_Ident_Intro_v23abr12.pdf
% convert from ft to ss https://www.mathworks.com/help/signal/ref/tf2ss.html
% convert from ss to ft https://www.mathworks.com/help/matlab/ref/ss2tf.html

Para entender a dinâmica de um sistema uma boa maneira é observando a saída ao aplicar uma entrada. As formas de ondas de entrada para uma melhor compreensão do funcionamento do sistema são ondas triangulares e quadradas. Entradas triangulares dão uma visão das não-linearidades dos sistema, ou seja, a partir de que valor de entrada é possível observar uma saída, isso acontece devido aos atritos dos engrenamentos. E entradas quadradas possibilitam a identificação dos parâmetros da função de transferência, como será demonstrado. Assim, a primeira parte nessa etapa é a identificação de não-linearidades para que na etapa seguinte seja possível identificar os parâmetros da função de transferência com maior precisão. A entrada e saída escalonadas da planta real podem ser observadas nas figuras abaixo onde a entrada é tensão e a saída velocidade angular.

\begin{figure}[H]
    \centering
    \includegraphics[width=1\linewidth]{tex/img/Entrada_e_Saida.jpg}
    \caption{Entrada e Saída do Sistema Real}
    \label{fig:entrada_saida_real}
\end{figure}

\begin{figure}[H]
    \centering
    \includegraphics[width=1\linewidth]{tex/img/Entrada_e_Saida_Linear.jpg}
    \caption{Entrada e Saída do Sistema Real}
    \label{fig:entrada_saida_real_linear}
\end{figure}

\subsection{Região Não-linear}
Idealmente para qualquer tanto de tensão aplicada, o motor responderia com uma velocidade proporcional. No entanto, como foi possível observar a partir do gráfico na figura \ref{fig:entrada_saida_real}, existe de fato uma não-linearidade na resposta a entrada. Esse efeito é chamado de Zona Morta ou Banda Morta e para o melhor funcionamento da dinâmica do sistema, deve ser evitado ou compensado. Para tanto, foi utilizada a seguinte planta genérica para validar o modelo de identificação da Zona Morta a partir de parâmetros conhecidos.

% Modelo Figura
\begin{figure}[H]
    \centering
    \includegraphics[width=1.0\linewidth]{tex/img/Sistema_Generico.png}
    \caption{Sistema Genérico de Validação da Parte Não-linear}
    \label{fig:sistema_generico_nlinear}
\end{figure}

Para identificar a Banda Morta, buscamos pelo o maior valor e o menor valor para a entrada tal que a saída seja zero. E assim remover o efeito desta região no controlador posteriormente.

\begin{equation}
    y_{dz}(x) = 
    \left\{\begin{array}{c}
    y(x), x > \delta_{max} \\
    y(x), x < \delta_{min} \\
    0, \delta_{-} < x < \delta_{+} \\
\end{array} \right.
\end{equation}

Segue o código usado para identificação:

% Código
\inputminted[frame=single,framesep=10pt]{matlab}{../src/matlab/deadzoneindetify.m}

% Maior valor da entrada pelo qual a saída é zero.

Com essa abordagem e para a planta genérica foi identificado \textit{delta+} com 0.03\%\ de erro \textit{delta-} com 0.16\%\ de erro. E já para a planta real os valores obtidos são: \textit{delta+} de 0.5635 e \textit{delta-} de -0.5296.

% Colocar Valores obtidos pelo matlab

\subsection{Estimando o Modelo como Sistema de Primeira Ordem}

Supondo um modelo de primeira ordem com coeficientes $k_1$ e $k_2$ definido por:

\begin{equation}\label{eq:firstordertf}
    H_1(s) = \frac{Y(s)}{X(s)} = \frac{1}{k_1 s+ k_2}
\end{equation}

% Equação de Transferência

Seja a função de transferência. Então, é possível reescrever essa função de transferência como:
\begin{equation}
k_1 Ys + k_2 Y  = X
\end{equation}

% EDO
De onde o sistema é descrito a partir da seguinte equação diferencial:

\begin{equation}
k_1 \dot{y}(t) + k_2 y(t) = x(t)
\end{equation}

Tomando os valores de entrada e saída do sistema em diferentes instantes para um sinal de entrada temos:

% Sistema Linear
\begin{equation}
\left\{\begin{array}{c}
    k_1 \dot{y}(t_0) + k_2 y(t_0) = x(t_0)  \\
    \dots\\
    k_1 \dot{y}(t_i) + k_2 y(t_i) = x(t_i)  \\
\end{array} \right.
\end{equation}

% Representação como Matriz

Assim, é possível representar a equação na seguinte forma matricial:

\begin{equation}
\left[\begin{array}{cc}
    \dot{y}(t_0) & y(t_0)\\
    \dots  & \dots \\
    \dot{y}(t_i) & y(t_i)\\
\end{array} \right]
\left[\begin{array}{c}
    k_1\\
    k_2\\
\end{array} \right]
=
\left[\begin{array}{c}
    x(t_0)  \\
    \dots\\
    x(t_i)  \\
\end{array} \right]
\end{equation}

% Resolução por Matrizes

Denominando $A$ a matriz envolvendo $\dot{y}(t)$ e $y(t)$, $Bt)$ a matriz dos valores de entrada $x(t)$

\begin{equation}
    A(t) P = B(t)
\end{equation}

\begin{equation}
    A^T(t)A(t) P = A^T(t) B(t)
\end{equation}

Isolando a matriz $P$ podemos achar o parâmetros a partir da seguinte expressão:

\begin{equation}
    P = \left(A^T(t) A(t)\right)^{-1} A^T(t) B(t)
\end{equation}

Para implementar em \textit{MATLAB} pode ser feita a aproximação de $\dot{y}(t)$ pelo método de diferenças finitas, como:

\begin{equation}
\dot{y}(t_i) = \dot{y}[t_i] = \frac{y[t_i] - y[t_{i-1}]}{t_i - t_{i-1}}
\end{equation}

Usando operações de matrizes o código fica bem compacto, como mostrado abaixo:

% Código
\inputminted[frame=single,framesep=10pt]{matlab}{../src/matlab/firstordertf.m}

\subsection{Estimando o Modelo como Sistema de Segunda Ordem}

Supondo um modelo de primeira ordem com coeficientes $k_1$, $k_2$ e $k_3$ definido por:

\begin{equation}\label{eq:secondordertf}
    H_2(s) = \frac{Y(s)}{X(s)} = \frac{1}{k_1 s^2+ k_2 s+ k_3}
\end{equation}

% Equação de Transferência

Podemos reescrever a função de transferência como
\begin{equation}
k_1 Ys^2 + k_2 Ys + k_3 Y  = X
\end{equation}

% EDO
A partir do qual temos o sistema descrito a partir da seguinte equação diferencial:

\begin{equation}
k_1 \ddot{y}(t) + k_2 \dot{y}(t) + k_3 y(t) = x(t)
\end{equation}

Tomando os valores de entrada e saída do sistema em diferentes instantes para um sinal de entrada temos:

% Sistema Linear
\begin{equation}
\left\{\begin{array}{c}
    k_1 \ddot{y}(t_0) + k_2 \dot{y}(t_0) + k_3 y(t_0) = x(t_0)  \\
    \dots\\
    k_1 \ddot{y}(t_i) + k_2 \dot{y}(t_i) + k_3 y(t_i) = x(t_i)  \\
\end{array} \right.
\end{equation}

% Representação como Matriz

Podemos representar a equação na forma matricial:

\begin{equation}
\left[\begin{array}{ccc}
    \ddot{y}(t_0) & \dot{y}(t_0) & y(t_0)\\
    \dots  & \dots & \dots \\
    \ddot{y}(t_i) & \dot{y}(t_i) & y(t_i)\\
\end{array} \right]
\left[\begin{array}{c}
    k_1\\
    k_2\\
    k_3\\
\end{array} \right]
=
\left[\begin{array}{c}
    x(t_0)  \\
    \dots\\
    x(t_i)  \\
\end{array} \right]
\end{equation}

% Resolução por Matrizes

De forma similar ao caso de primeira ordem, denominando $A$ a matriz envolvendo $\ddot{y}(t)$, $\dot{y}(t)$ e $y(t)$, $Bt)$ a matriz dos valores de entrada $x(t)$

\begin{equation}
    A(t) P = B(t)
\end{equation}

\begin{equation}
    A^T(t)A(t) P = A^T(t) B(t)
\end{equation}

Isolando a matriz $P$ podemos achar o parâmetros a partir da seguinte expressão:

\begin{equation}
    P = \left(A^T(t) A(t)\right)^{-1} A^T(t) B(t)
\end{equation}

Em MATLAB, usando operações de matrizes a implementação fica bem compacta, como mostrado abaixo:

% Código
\inputminted[frame=single,framesep=10pt]{matlab}{../src/matlab/secondordertf.m}

\subsection{Validação do Método de Identificação em Simulação}

Aplicando um sinal de entrada em uma planta simulada de primeira ordem com coeficientes escolhidos de forma aleatória, obtemos a seguinte as seguintes respostas para o modelo aproximado de primeira ordem e de segunda ordem. Em que $m_1$ é a resposta do modelo estimado de primeira ordem e $m_2$ é a do modelo de segunda ordem.

% Modelo Figura
Procedendo da mesma forma, aplicando um sinal de entrada em uma planta simulada de segunda ordem com coeficientes escolhidos de forma aleatória, são obtidas as seguintes respostas para o modelo aproximado de primeira ordem e de segunda ordem. Conforme mostrado na figura \ref{fig:modelEvaluation} em que compara a estimativa obtida pela identificação por ambos métodos a partir dos dados de simulação de um sistema de primeira ordem e de segunda ordem.

\begin{figure}[H]
    \centering
    \begin{subfigure}[b]{0.5\textwidth}
        \centering
        \includegraphics[width=0.8\linewidth,trim={0 2cm 0 0}]{tex/img/model1Evaluation.pdf}
        \caption{Modelo Simulado de Primeira Ordem}
    \end{subfigure}%
    ~ 
    \begin{subfigure}[b]{0.5\textwidth}
        \centering
        \includegraphics[width=0.8\linewidth,trim={0 2cm 0 0}]{tex/img/model2Evaluation.pdf}
        \caption{Modelo Simulado de Segunda Ordem}
    \end{subfigure}%
    \caption{Comparação Resposta Modelo identificado de primeira e segunda ordem vs simulado}
    \label{fig:modelEvaluation}
\end{figure}

Nota-se que em ambos casos, existe um erro entre a resposta do sistema com os parâmetros identificados e o planta original simulada. Isto se deve ao fato que o método de diferenças finitas apenas representa uma aproximação e portanto não representa derivada de forma exata.

Em uma planta real pode ser utilizado algum sensor para medir a velocidade e a aceleração ou ainda adotar alguma aproximação melhor para aproximação da derivada. A partir dos resultados o modelo de identificação que apresentou melhores resultados foi o modelo de primeira ordem, portanto, é o aplicado nesse trabalho.

% Descrição da Planta
%% Diagrama de descrição

\subsection{Resultados Obtidos da Planta Real}

Uma vez validado o método e aplicando para os dados de entrada e saída reais, foram obtidos os seguintes valores para os coeficientes, conforme mostrado na tabela \ref{tab:tf}.

\begin{table}[H]
    \centering
    \begin{tabular}{r|cc}
    \hline
        Modelo & $s^1$ & $s^0$ \\
     \hline
        Primeira Ordem & 0.0035 & 0.0354 \\
     \hline
    \end{tabular}
    \caption{Valores dos coeficientes obtidos pela identificação}
    \label{tab:tf}
\end{table}

% ---------------------------------------------------------------------------------------

\section{Controle Proporcional-Integrativo-Derivativo}

% Comentar brevemente sobre a estrutura de PID usada
% Anti Windup

Para projeto de um sistema de controle para a posição angular do sistema rotativo, foi usado um controlador do tipo PID, que na forma paralela tem o seguinte formato:

\begin{equation}
    G_c(s) = K_p + Kd s + \frac{Ki}{s}
\end{equation}

No entanto, o controle PID sozinho não tem um bom desempenho quando o atuador satura. Quando isso acontece o termo integrativo do controle continua atuando e assim gera bastante sobressinal e até saídas cada vez mais crescentes, podendo danificar a planta. Para resolver esse problema foi implementado a estrutura do PID com Anti-windup, que em análise simplificada, bloqueia o termo integrativo nos momentos em que não seja necessário.

%http://www.cds.caltech.edu/~murray/books/AM08/pdf/am06-pid_16Sep06.pdf

\subsection{Estrutura da Planta}

A partir dos dados da identificação, a planta com o controle PID implementado está ilustrada na figura \ref{fig:planta_PID}.

\begin{figure}[H]
    \centering
    \includegraphics[width=1.0\linewidth]{tex/img/Planta_PID.png}
    \caption{Planta Simulação Controle PID}
    \label{fig:planta_PID}
\end{figure}
A função de transferência como é apresentada na figura acima, pode ser obtida de:

\begin{equation}\label{eq:firstordertf1}
    H_1(s) = \frac{Y(s)}{X(s)} = \frac{K}{k_1 s+ k_2}
\end{equation}

De onde se obtém o ganho como:

\begin{equation}
    \left\{\begin{array}{c}
    K = \frac{1}{k_2} \\
    k_1 = Kk_2 \\
\end{array} \right.
\end{equation}

A função inversa é dada a partir dos parâmetros \textit{delta+} e \textit{delta-}, e tem a seguinte estrutura:

\begin{equation}
    y(x) = 
    \left\{\begin{array}{c}
    x = x + 0.5635, x > 0 \\
    x = x - 0.5296, x < 0 \\
    x, x = 0 \\
\end{array} \right.
\end{equation}

\subsection{Ajuste dos Parâmetros}

Existem diversas formas para ajustar os parâmetros de um controlador PID, mas primeiro é necessário definir os critérios de projeto. Neste projeto foi objetivado o equilíbrio entre o tempo de resposta e o \textit{Overshoot}, ou seja, a falta dele. Para tal, foi utilizada a ferramenta \textit{Sisotool} do \textit{MATLAB} para definir os ganhos do controlador. A partir deste procedimento foram definidos os seguintes valores para os ganhos, registrados na tabela \ref{tab:pid_param}

\begin{table}[H]
    \centering
    $$\begin{array}{cccc}
    \hline
        K_p & K_i & K_d & Ganho Anti-windup \\
    \hline
        0.11168 & 0.00418 & 0.00268 & \sqrt{0.00418*0.00268}\\
     \hline
    \end{array}$$
    \caption{Parâmetros do Controlador PID}
    \label{tab:pid_param}
\end{table}

Na figura abaixo consta a resposta simulada do sistema à uma entrada degrau com e sem a aplicação do controlador.

\begin{figure}[H]
    \centering
    \includegraphics[width=0.6\linewidth]{tex/img/Resposta_ao_degrau.jpg}
    \caption{Resposta à uma Entrada Degrau}
    \label{fig:resposta_ao_degrau}
\end{figure}

% Foto SISO Tool

\subsection{Implementação na Planta Real}

Uma vez simulado e testado, o controlador projetado foi avaliado na planta real. Foi notado uma leve diferença na resposta, o que é justificável, uma vez que o modelo obtido é uma aproximação do modelo real. Porém a resposta na planta real satisfez por completo os critérios de projeto, conforme mostrado na figura \ref{fig:quaser_pid}.

\begin{figure}[H]
    \centering
    \includegraphics[width=0.8\linewidth]{tex/img/quanserpid_s90num5.png}
    \caption{Resposta Planta Real PID}
    \label{fig:quaser_pid}
\end{figure}

Em seguida foram avaliados a resposta do mesmo controlador para outras condições de frequência de entrada e amplitude do sinal. Para uma frequência de $1Hz$ e um ângulo de $45^\circ$ foi notado que o controlador não conseguiu acompanhar a referência, como é ilustrado na figura \ref{fig:quanserpid_s45num1}.

\begin{figure}[H]
    \centering
    \includegraphics[width=0.8\linewidth]{tex/img/quanserpid_s45num1.png}
    \caption{Resposta Planta Real PID 45 graus}
    \label{fig:quanserpid_s45num1}
\end{figure}

No entanto, para o experimento com um valor grande de variação como $180^\circ$ o controlador manteve um bom desempenho, ainda que o \textit{Overshoot} do sistema tenha aumentado. Houve um deslocamento referenciado positivo maior que no negativo, como mostrado na figura \ref{fig:quanserpid_s180num5}.

\begin{figure}[H]
    \centering
    \includegraphics[width=0.8\linewidth]{tex/img/quanserpid_s180num5.png}
    \caption{Resposta Planta Real PID 180 graus}
    \label{fig:quanserpid_s180num5}
\end{figure}

% ---------------------------------------------------------------------------------------
\section{Controle em Espaço de Estados}

Foi avaliado também o controle por alocação de polos utilizando a representação no espaço de estados. Esta forma de representação permite um posicionamento mais livre dos polos em malha fechada garantindo um melhor desempenho do sistema.

\begin{equation}
\left\{
\begin{array}{c}
    \dot{X} = A X + B U \\
    Y = C X + D U \\
\end{array}
\right.
\end{equation}

A figura \ref{fig:space_state_model} apresenta a representação de um sistema genérico em espaço de estados.

% Espaço de estados
\begin{figure}[H]
    \centering
    \includegraphics{tex/img/space_state_model.png}
    \caption{Estrutura da Representação em Espaço de Estados}
    \label{fig:space_state_model}
\end{figure}

\subsection{Estrutura da Planta}

Como o modelo foi identificado a partir de uma função de transferência é necessário converter para a representação em espaço de estados. Para um dado sistema existem várias representações possíveis.

Para o caso o sistema de primeira ordem descrito em \ref{eq:firstordertf}, fica:

$$
G_1(s) = H_1(s)\cdot \frac{1}{s} = \frac{1}{k_1 s^2+ k_2 s} = \frac{1}{s^2+ \frac{k_2}{k_1} s}
$$

A partir do qual é possível escrever a matriz na forma controlada como:

\begin{equation}
\left\{
\begin{array}{rcl}
\left[\begin{array}{c}
    \dot{X_1} \\
    \dot{X_2} \\
\end{array}
\right]
&
=
&
\left[
\begin{array}{cc}
    0 & 1 \\
    -\frac{k_2}{k_1} & 0 \\ 
\end{array}
\right]
\left[\begin{array}{c}
    X_1 \\
    X_2 \\
\end{array}
\right]
+
\left[\begin{array}{c}
    0 \\
    1 \\
\end{array}
\right]
U\\
Y & = &\left[
\begin{array}{cc}
    1 & 0 \\
\end{array}
\right]
\left[\begin{array}{c}
    X_1 \\
    X_2 \\
\end{array}
\right]
\end{array}
\right.
\end{equation}

De forma similar para o caso do sistema de segunda ordem descrito em \ref{eq:secondordertf}, fica:

$$
G_2(s) = H_2(s)\cdot \frac{1}{s} = \frac{1}{k_1 s^3 + k_2 s^2 + k_3 s} = \frac{1}{s^3 + \frac{k_2}{k_1} s^2 + \frac{k_3}{k_1} s}
$$

A partir do qual é possível escrever a matriz na forma controlada como:

\begin{equation}
\left\{
\begin{array}{rcl}
\left[\begin{array}{c}
    \dot{X_1} \\
    \dot{X_2} \\
    \dot{X_3} \\
\end{array}
\right]
&
=
&
\left[
\begin{array}{ccc}
    0 & 1 & 0 \\
    0 & 0 & 1 \\
    -\frac{k_2}{k_1} & -\frac{k_3}{k_1} & 0 \\ 
\end{array}
\right]
\left[\begin{array}{c}
    X_1 \\
    X_2 \\
    X_3 \\
\end{array}
\right]
+
\left[\begin{array}{c}
    0 \\
    0 \\
    1 \\
\end{array}
\right]
U\\
Y & = &\left[
\begin{array}{ccc}
    1 & 0 & 0 \\
\end{array}
\right]
\left[\begin{array}{c}
    X_1 \\
    X_2 \\
    X_3 \\
\end{array}
\right]
\end{array}
\right.
\end{equation}

No \textit{MATLAB}, foi utilizada a função \textit{$tf2ss$} que converte direto da representação como função de transferência para espaço de estados. Embora seja prático, a matriz final não possui interpretação física dos valores. Assim, as matrizes do sistema são:

\begin{equation}
A = \left[\begin{array}{cc}
    -10 & 0 \\
    1 & 0   \\
\end{array}\right]
B = \left[\begin{array}{c}
    1 \\
    0 \\
\end{array}
\right]
C = \left[\begin{array}{cc}
    0 & 282.20\\
\end{array}
\right]
D = \left[\begin{array}{c}
    0 \\
\end{array}
\right]
\end{equation}

A estrutura completa do sistema modelado em espaço de estados é apresentada a seguir:

\begin{figure}[H]
    \centering
    \includegraphics[width=0.8\linewidth]{tex/img/Planta_Espaco_de_Estados.png}
    \caption{Planta Completa em Espaço de Estados}
    \label{fig:Planta_Espaco_de_Estados_Completa}
\end{figure}

\subsection{Ajuste dos Parâmetros}

A representação por espaço de estados traz a vantagem de permitir posicionar os polos em malha fechada de maneira arbitrária facilitando atender qualquer critério de projeto.

De forma similar foi feito no \textit{MATLAB} com ajuda do comando \textit{solve} para resolução de equações simbólicas, conforme mostrado no código abaixo:

\inputminted[frame=single,framesep=10pt]{matlab}{../src/matlab/Fase-3/Script_Controle_Espaco_de_Estados.m}

Com o código apresentado foi possível obter os demais parâmetros da planta \ref{fig:Planta_Espaco_de_Estados_Completa}, como apresentados abaixo:

\begin{equation}
K_i = \left[\begin{array}{c}
    2.552 \\
\end{array}\right]
K_e = \left[\begin{array}{c}
    0.0071 \\
    0.0248 \\
\end{array}
\right]
K = \left[\begin{array}{cc}
    17 & 17 242 \\
\end{array}
\right]
\end{equation}

\subsection{Implementação na Planta Real}

Para valores de ângulos menores o controlador em espaço de estados implementado obteve uma resposta mais rápida que o PID, permitindo o uso de frequências maiores como demonstrado na figura \ref{fig:quanserss_s10num1} e \ref{fig:quanserss_s45num1}.

\begin{figure}[H]
    \centering
    \includegraphics[width=0.8\linewidth]{tex/img/quanserss_s10num1.png}
    \caption{Experimento SS 10 graus}
    \label{fig:quanserss_s10num1}
\end{figure}

\begin{figure}[H]
    \centering
    \includegraphics[width=0.8\linewidth]{tex/img/quanserss_s45num1.png}
    \caption{Experimento SS 45 graus}
    \label{fig:quanserss_s45num1}
\end{figure}


No entanto, para um valor alto de para o ângulo de referência, o controlador gerou um sobressinal. O que pode ser explicado pela saturação do motor. Tendo que no controlador em espaço de estados funciona a partir de um conjunto de integradores e como não foi implementado nenhuma forma \textit{Anti-windup} o erro foi acumulado com a saturação da velocidade gerando um sobressinal.

\begin{figure}[H]
    \centering
    \includegraphics[width=0.8\linewidth]{tex/img/quanserss_s90num5.png}
    \caption{Experimento SS 90 graus}
    \label{fig:quanserss_s90num5}
\end{figure}

Notadamente, ao solicitar uma velocidade alta para controlador, por meio de uma sinal de referência com $\theta = 180^\circ$ e frequência $f = 1Hz$ o controlador não conseguiu acompanhar a referência, conforme mostrado na figura \ref{fig:quanserss_s180num1}.

\begin{figure}[H]
    \centering
    \includegraphics[width=0.8\linewidth]{tex/img/quanserss_s180num1.png}
    \caption{Experimento SS 180 graus}
    \label{fig:quanserss_s180num1}
\end{figure}

\section{Comparando o Desempenho dos Controladores}

Comparando o desempenho dos dois controladores temos, que o controlador em espaço de estados possui uma resposta mais rápida como mostrado em \ref{fig:control_s45dt1}.

\begin{figure}[H]
    \centering
    \begin{subfigure}[b]{0.5\textwidth}
        \centering
        \includegraphics[width = \linewidth]{tex/img/quanserpid_s45num1.png}
        \caption{PID}
    \end{subfigure}%
    ~ 
    \begin{subfigure}[b]{0.5\textwidth}
        \centering
        \includegraphics[width = \linewidth]{tex/img/quanserss_s45num1.png}
        \caption{Espaço de Estados}
    \end{subfigure}%
    \caption{Resposta Controladores para $\theta=45^\circ$ e $f = 1Hz$}
    \label{fig:control_s45dt1}
\end{figure}

No entanto, a ausência de alguma forma de \textit{Anti-windup} gera um sobressinal para o controle no espaço de estados, tal como apresentado na figura \ref{fig:control_s90dt5}.

\begin{figure}[H]
    \centering
    \begin{subfigure}[b]{0.5\textwidth}
        \centering
        \includegraphics[width = \linewidth]{tex/img/quanserpid_s90num5.png}
        \caption{PID}
    \end{subfigure}%
    ~ 
    \begin{subfigure}[b]{0.5\textwidth}
        \centering
        \includegraphics[width = \linewidth]{tex/img/quanserss_s90num5.png}
        \caption{Espaço de Estados}
    \end{subfigure}%
    \caption{Resposta Controladores para $\theta=90^\circ$ e $f = 5Hz$}
    \label{fig:control_s90dt5}
\end{figure}

Para um angulo maior de entrada, o sobressinal é ainda mais perceptível, denotando o impacto da saturação do motor na velocidade do controlador. Como efeito, nesta situação o erro é pior para o controlador em \ref{fig:control_s180dt5}. Nesta situação o controlador implementado em espaço de estados nem chega a estabilizar em decorrência do enorme sobressinal.

\begin{figure}[H]
    \centering
    \begin{subfigure}[b]{0.5\textwidth}
        \centering
        \includegraphics[width = \linewidth]{tex/img/quanserpid_s180num5.png}
        \caption{PID}
    \end{subfigure}%
    ~ 
    \begin{subfigure}[b]{0.5\textwidth}
        \centering
        \includegraphics[width = \linewidth]{tex/img/quanserss_s180num5.png}
        \caption{Espaço de Estados}
    \end{subfigure}%
    \caption{Resposta Controladores para $\theta=180^\circ$ e $f = 5Hz$}
    \label{fig:control_s180dt5}
\end{figure}

Observa-se também a alta variação na posição durante o pico o que mostra que o controlador deixou de comportar a dinâmica do sistema, fornecendo a maior tensão para a planta para a subida e para a decida. O que mostra que o controlador está operando fora da faixa linear do sistema.

% ---------------------------------------------------------------------------------------

\section{Conclusão}

Através de ambos projetos foi possível compreender o funcionamento de dois tipos de controladores diferentes e comumente encontrados, o PID e controle baseado em espaço de estados. Embora o controle PID seja mais simples de implementar em sistemas analógicos o controle por alocação de polos no sistema em espaço de estados se mostrou mais simples de ajustar e implementar, uma vez que todas as contas podem ser feitas diretamente no \textit{MATLAB}. O que reduziu bastante o esforço no ajuste dos parâmetros.

Ainda com as aproximações usadas, o método de identificação usado conduziu a um resultado satisfatório dentro dos objetivos pretendidos. Tendo sido notado pequenas variações na resposta de ambos controladores implementados na planta real em relação a simulação. Em particular a resposta no espaço de estados apresentou um leve sobressinal (1\%) quando foi projetado uma resposta amortecida. Enquanto para o PID o desempenho real foi melhor que em simulação e ainda maior controlabilidade sobre o ângulo de rotação do sistema, ou seja, comportando maiores variações.

% ---------------------------------------------------------------------------------------

\bibliographystyle{abbrv}
\bibliography{references}
% Referências
% Acrescentadas no arquivo references.bib
% para usa-las no texto basta usar \citep{}

\nocite{ogata2010modern}

% ---------------------------------------------------------------------------------------

\newpage

\begin{center}
    \huge \textbf{Laboratório de Controle Dinâmico \\}
\end{center}

\begin{table}[H]
    \centering
    \begin{tabular}{|c|p{6cm}|c|}
    \hline
         & Atribuições & Assinatura \\
    \hline
                        & 1. Identificação do Modelo & \\
    Tiago Pereira Neves & 2. Projeto do Controlador PID & \\
                        & 3. Projeto do Controlador SS & \\
    \hline

                & 1. Identificação do Modelo &\\
    Rafael Lima & 2. Formatação Relatório &\\
                & 3. Análise de Dados Experimentais& \\
     \hline
    \end{tabular}
\end{table}

\thispagestyle{empty}

% ---------------------------------------------------------------------------------------


\end{document}